\documentclass{beamer} %
\usetheme{CambridgeUS}
\usepackage[slovene]{babel}  % slovenščina
\usepackage[T1]{fontenc}     % naprednejše kodiranje fonta
\usefonttheme{professionalfonts}
\usepackage{times}
\usepackage{tikz}
\usepackage{amsmath}
\usepackage{verbatim}
\usetikzlibrary{arrows,shapes}

\author{Matic Oskar Hajšen in Eva Zmazek}
\title{Gibanja togih teles}

% ukazi za matematična okolja
\theoremstyle{definition} % tekst napisan pokončno
\newtheorem{definicija}{Definicija}[section]
\newtheorem{primer}[definicija]{Primer}
\newtheorem{opomba}[definicija]{Opomba}
\newtheorem{aksiom}{Aksiom}

\theoremstyle{plain} % tekst napisan poševno
\newtheorem{lema}[definicija]{Lema}
\newtheorem{izrek}[definicija]{Izrek}
\newtheorem{trditev}[definicija]{Trditev}
\newtheorem{posledica}[definicija]{Posledica}

\numberwithin{equation}{section}  % števec za enačbe zgleda kot (2.7) in se resetira v vsakem poglavju

% Matematični ukazi
\newcommand{\R}{\mathbb R}
\newcommand{\N}{\mathbb N}
\newcommand{\Z}{\mathbb Z}
%\renewcommand{\C}{\mathbb C}
\newcommand{\Q}{\mathbb Q}
\renewcommand{\H}{\mathbb H}

\begin{document}



\begin{frame}
\titlepage
\end{frame}

\begin{frame}
\frametitle{Homogene in kartezične koordinate}

$P$ vektor v $3$-dimenzionalnem prostoru $\R^3$\\

\begin{itemize}
\item Homogene koordinate vektorja $P$: $$p = (p_0,p_1,p_2,p_3)^T \in \R^4 / \{(0,0,0,0)^T\}$$
\item Če $p_0 \not= 0$, potem so kartezične koordinate vektorja $P$ enake $$\underline{p} = (\underline{p_1}, \underline{p_2}, \underline{p_3})^T \in \R^3;~  \underline{p_i} = \frac{p_i}{p_0} \text{ za } i=1,2,3$$
\item Vektorjem z ničelno prvo homogeno komponento priredimo točke v neskončnosti.
\end{itemize}

\begin{opomba}
Vektorja $p$ in $\lambda p$ v homogenih koordinatah opisujeta isti vektor $\underline{p}$ v kartezičnih koordinatah za poljubno neničelno realno število $\lambda$.
\end{opomba}

\end{frame}

\begin{frame}
\frametitle{Fiksen in gibajoč se koordinatni sistem}

Definirajmo dva koordinatna sistema v $\R^3$:
\begin{itemize}
\item fiksen koordinatni sistem $E^3$ (običajen koordinatni sistem)
\item gibajoč se koordinatni sistem $\hat{E}^3$
\end{itemize}
Točke lahko predstavimo v enem ali drugem.\\

\end{frame}

\begin{frame}
Potrebujemo koordinatno transformacijo
$$\hat{E}^3 \to E^3$$
$$\hat{p} \mapsto p$$


$$M =
\left[
\begin{tabular}{c | c c c}
$m_{0,0}$ & $0$ & $0$ & $0$ \\
\hline
$m_{1,0}$ & $m_{1,1}$ & $m_{1,2}$ & $m_{1,3}$ \\
$m_{2,0}$ & $m_{2,1}$ & $m_{2,2}$ & $m_{2,3}$ \\
$m_{3,0}$ & $m_{3,1}$ & $m_{3,2}$ & $m_{3,3}$ \\
\end{tabular}
\right],
$$

$$\hat{p} \mapsto p = M \hat{p}$$

\end{frame}

\begin{frame}
\frametitle{Transformacija koordinatnega izhodišča}

Koordinatno izhodišče:
\begin{itemize}
\item v kartezičnih koordinatah: $(0,0,0)^T$
\item v homogenih koordinatah: $(1,0,0,0)^T$
\end{itemize}

\vspace{1cm}
Transformirano koordinatno izhodišče:
\begin{itemize}
\item v homogenih koordinatah:
$$c = M (1,0,0,0)^T = (m_{0,0}, m_{1,0}, m_{2,3}, m_{3,0})^T$$
\item v kartezičnih koordinatah:
$$\underline{c} = (\frac{m_{1,0}}{m_{0,0}}, \frac{m_{2,0}}{m_{0,0}}, \frac{m_{3,0}}{m_{0,0}})^T$$
\end{itemize}

\end{frame}

\begin{frame}
\frametitle{Rotacijska matrika}

$$\underline{R} =
\frac{1}{m_{0,0}}
\left[
\begin{tabular}{c c c}
$m_{1,1}$ & $m_{1,2}$ & $m_{1,3}$ \\
$m_{2,1}$ & $m_{2,2}$ & $m_{2,3}$ \\
$m_{3,1}$ & $m_{3,2}$ & $m_{3,3}$ \\
\end{tabular}
\right]
$$

\noindent opisuje orientacijo gibajočega se koordinatnega sistema $\hat{E}^3$
%\noindent $\underline{c}$ predstavlja položaj izhodišča koordinatnega sistema $\hat{E}^3$ v koordinatni sistem $E^3$, $R$ pa je \textbf{rotacijska matrika}, ki opisuje rotacijo gibajočega se koordinatnega sistema. \\

\end{frame}

\begin{frame}
\frametitle{Transformacija vektorja $[1,b_M,c_M,d_M]$}


$$d = 
M
\cdot
\left[
\begin{tabular}{c}
$1$ \\
$b_M$ \\
$c_M$ \\
$d_M$
\end{tabular}
\right]
=
\left[
\begin{tabular}{c}
$m_{0,0}$ \\
$m_{1,0}$ \\
$m_{2,0}$ \\
$m_{3,0}$ \\
\end{tabular}
\right]
+
\left[
\begin{tabular}{c c c}
$0$ & $0$ & $0$ \\
$m_{1,1} b_M$ & $m_{1,2} c_M$ & $m_{1,3} d_M$ \\
$m_{2,1} b_M$ & $m_{2,2} c_M$ & $m_{2,3} d_M$ \\
$m_{3,1} b_M$ & $m_{3,2} c_M$ & $m_{3,3} d_M$ \\
\end{tabular}
\right]
$$

$$d_0 = m_{0,0}$$
$$\left[
\begin{tabular}{c}
$d_1$ \\
$d_2$ \\
$d_3$ \\
\end{tabular}
\right]
=
\left[
\begin{tabular}{c}
$m_{1,0}$ \\
$m_{2,0}$ \\
$m_{3,0}$ \\
\end{tabular}
\right]
+
\left[
\begin{tabular}{c c c}
$m_{1,1}$ & $m_{1,2}$ & $m_{1,3}$ \\
$m_{2,1}$ & $m_{2,2}$ & $m_{2,3}$ \\
$m_{3,1}$ & $m_{3,2}$ & $m_{3,3}$ \\
\end{tabular}
\right]
\cdot
\left[
\begin{tabular}{c}
$b_M$ \\
$c_M$ \\
$d_M$
\end{tabular}
\right]
$$

\end{frame}

\begin{frame}


$$d_0 = m_{0,0}$$
$$\left[
\begin{tabular}{c}
$d_1$ \\
$d_2$ \\
$d_3$ \\
\end{tabular}
\right]
=
\left[
\begin{tabular}{c}
$m_{1,0}$ \\
$m_{2,0}$ \\
$m_{3,0}$ \\
\end{tabular}
\right]
+
\left[
\begin{tabular}{c c c}
$m_{1,1}$ & $m_{1,2}$ & $m_{1,3}$ \\
$m_{2,1}$ & $m_{2,2}$ & $m_{2,3}$ \\
$m_{3,1}$ & $m_{3,2}$ & $m_{3,3}$ \\
\end{tabular}
\right]
\cdot
\left[
\begin{tabular}{c}
$b_M$ \\
$c_M$ \\
$d_M$
\end{tabular}
\right]
$$

$$\underline{d}
=
\frac{1}{m_{0,0}}\left[
\begin{tabular}{c}
$m_{1,0}$ \\
$m_{2,0}$ \\
$m_{3,0}$ \\
\end{tabular}
\right]
+
\frac{1}{m_{0,0}}
\left[
\begin{tabular}{c c c}
$m_{1,1}$ & $m_{1,2}$ & $m_{1,3}$ \\
$m_{2,1}$ & $m_{2,2}$ & $m_{2,3}$ \\
$m_{3,1}$ & $m_{3,2}$ & $m_{3,3}$ \\
\end{tabular}
\right]
\cdot
\left[
\begin{tabular}{c}
$b_M$ \\
$c_M$ \\
$d_M$
\end{tabular}
\right]=
$$

$$=\underline{c} + R \cdot \underline{\hat{p}}$$

\noindent Transformacijo $\underline{\hat{p}} \mapsto \underline{p}$ v kartezičnih koordinatah zapišemo kot:
$$\underline{p} = \underline{c} + R \underline{\hat{p}}$$

\end{frame}

\begin{frame}
\frametitle{Gibanje točk v času}

\noindent Kadar je $\underline{c} = \underline{c}(t)$ in $R=R(t)$, govorimo o gibanju togega telesa:
$$\hat{E}^3 \times I \to E^3$$
$$(\underline{\hat{p}},t) \mapsto \underline{c}(t) + R(t) \underline{\hat{p}} =: \underline{p}(t)$$
Krivulji $\underline{p}(t)$ pravimo \textbf{trajektorija} točke $\underline{\hat{p}}$ \\

\noindent Če je $\underline{c}(t) = (0,0,0)$, potem trajektorija poljubne točke $\underline{\hat{p}}$ leži na sferi z radijem $||\underline{\hat{p}}||$ in središčem v koordinatnem izhodišču fiksnega koordinatnega sistema $E^3$.
\noindent Rotacijski del gibanja $R(t)$ opisuje gibanje po enotski sferi, zato se imenuje tudi \textbf{sferični del gibanja togega telesa}. Problem je konstrukcija matrike $R$, ki mora biti ortogonalna. ($R R^T = R^T R = I$, $\det R = 1$). \\

\end{frame}

\begin{frame}
\frametitle{Opis rotacij s kvaternioni}
\noindent Pri opisovanju rotacij si lahko pomagamo s \textbf{kvaternioni}. Prostor kvaternionov $\H$ je $4$-dimenzionalni vektorski prostor s standardno bazo
$$ \underline{1} = (1, (0,0,0)^T) $$
$$ \underline{i} = (0,(1,0,0)^T) $$
$$ \underline{j} = (0,(0,1,0)^T) $$
$$ \underline{k} = (0,(0,0,1)^T) $$

\noindent Vsak kvaternion $\mathcal{A}$ lahko zapišemo kot:
$$\mathcal{A} = (a_0,\underline{a}),~ a_0 \in \R \text{ skalarni del },~ \underline{a} = (a_1,a_2,a_3)^T \text{ vektorski del}$$

\end{frame}

\begin{frame}
\noindent Na kvaternionih sta definirana seštevanje in množenje kot:
$$\mathcal{A} + \mathcal{B} = (a_0,\underline{a}) +(b_0,\underline{b}) = (a_0 + b_0,\underline{a} + \underline{b})$$
$$\mathcal{A} \cdot \mathcal{B}  = (a_0 \cdot b_0 - \underline{a} \cdot \underline{b}, a_0 \underline{b} + b_0 \underline{a} + \underline{a} \times \underline{b})$$

\noindent Konjugirana vrednost kvaretniona $\mathcal{A} = (a_0,\underline{a})$ je definirana kot $\overline{\mathcal{A}} = (a_0, - \underline{a})$. S pomočjo konjugirane vrednosti nato definiramo tudi normo kvaterniona kot
$$|| \mathcal{A} || = \sqrt{\mathcal{A} \cdot \overline{\mathcal{A}}} = a_0^2 + a_1^2 + a_2^2 + a_3^2$$

\begin{definicija}
\label{definicijarotacije}
Preslikava $\chi : \H \backslash \{0\} \to SO_3$ oblike

$$Q  \mapsto \frac{1}{q_0^2 + q_1^2 + q_2^2 + q_3^2}
\left[
\begin{tabular}{c c c}
$q_0^2 + q_1^2 - q_2^2 - q_3^2$ & $2(q_1 q_2 - q_0 q_3)$ & $2(q_1 q_3 + q_0 q_2)$ \\
& & \\
$2(q_1 q_2 + q_0 q_3)$ & $q_0^2 - q_1^2 + q_2^2 - q_3^2$ & $2(q_2 q_3 - q_0 q_1)$ \\
& & \\
$2(q_1 q_3 - q_0 q_2)$ & $2(q_2 q_3 + q_0 q_2)$ & $q_0^2 - q_1^2 - q_2^2 + q_3^2$
\end{tabular}
\right]
$$

$$ Q = (q_0,(q_1,q_2,q_3)^T)$$
\\
se imenuje \textbf{kinematična preslikava}.
\end{definicija}

\end{frame}

\begin{frame}
\noindent Matirka $\chi(Q)$ je rotacijska matrika. Velja pa tudi obratno. Vsako rotacijsko matriko $R$ lahko zapišemo v zgornji obliki, to je, lahko jo preslikamo v dva \textbf{antipodna kvaterniona} oblike
$$\pm Q = \pm(q_0,(q_1,q_2,q_3)^T),$$
$$q_0^2 + q_1^2 + q_2^2 + q_3^2 = 1$$

\noindent \textbf{Kinematična preslikava} poda korespondenco med 3D rotacijami in parom antipodnih točk na 4D enotski sferi $S^3 \subseteq R^4$. \\

\end{frame}

\begin{frame}
\noindent Ker velja $q_0^2 + q_1^2 + q_2^2 + q_3^2 = 1$, so vrednosti $|q_i|;~ i=0,1,2,3$ na zaprtem interalu med $0$ in $1$. Vrednost $q_0$ in vektor $(q_1,q_2,q_3)^T$ lahko zato zapišemo v obliki:

$$q_0 = \cos(\frac{\phi}{2})$$

in

$$\left[
\begin{tabular}{c}
$q_1$ \\
$q_2$ \\
$q_3$
\end{tabular}
\right]
=
\sin(\frac{\phi}{2}) \cdot \vec{r}; ~ \vec{r} \text{ enotski vektor}
$$

\noindent Če kvaternion $Q$ zapišemo v tej obliki, ima rotacija, prirejena temu kvaternionu lepo geometrijsko interpretacijo. Predstavlja namreč rotacijo za kot $\phi$ okrog osi $\vec{r}$. \\

\end{frame}

\begin{frame}

\noindent Ker lahko vsako rotacijo zapišemo v tej obliki, lahko tako zapišemo tudi rotacijo iz poglavja \ref{zvezaFiksnikoor-Gibajockoor}. Če imamo podano preslikavo $M$, poiščimo, kako za to preslikavo definiramo kvaternion $Q$. Priemerjajmo matriki $\R$ in poglavja \ref{zvezaFiksnikoor-Gibajockoor} in $\R$, zapisanega s kvaternioni.

\begin{align*}
m_{0,0} + m_{1,1} + m_{1,2} + m_{3,3} &= 4 q_0^2 \\
m_{3,2} - m_{2,3} = 2 \cdot (q_2 q_3 - q_0 q_1) - 2 \cdot (q_2 q_3 + q_0 q_1) &= 4 q_0 q_1 \\
m_{1,3} - m_{3,1} = 2 \cdot (q_1 q_3 + q_0 q_2) - 2 \cdot (q_1 q_3 + q_0 q_2) &= 4 q_0 q_2 \\
m_{2,1} - m_{1,2} = 2 \cdot (q_1 q_2 + q_0 q_4) - 2 \cdot (q_1 q_2 + q_0 q_4) &= 4 q_0 q_3 \\
& \\
m_{3,2} - m_{2,3} & = 4 q_0 q_1 \\
m_{0,0} + m_{1,1} - m_{1,2} - m_{3,3} &= 4 q_1^2 \\
m_{2,1} + m_{1,2} &= 4 q_1 q_2 \\
m_{1,3} + m_{3,1} &= 4 q_1 q_3 \\
\end{align*}

\end{frame}

\begin{frame}
\begin{align*}
m_{1,3} - m_{3,1} &= 4 q_0 q_2 \\
m_{2,1} + m_{1,2} &= 4 q_1 q_2 \\
m_{0,0} - m_{1,1} + m_{1,2} - m_{3,3} &= 4 q_2^2 \\
m_{3,2} + m_{2,3} &= 4 q_2 q_3 \\
& \\
m_{2,1} - m_{1,2} &= 4 q_0 q_3 \\
m_{1,3} + m_{3,1} &= 4 q_1 q_3 \\
m_{3,2} + m_{2,3} &= 4 q_2 q_3 \\
m_{0,0} - m_{1,1} - m_{1,2} + m_{3,3} &= 4 q_0^2 \\
\end{align*}

\noindent Opazimo, da v vsakem sklopu razmerja med vrednostmi enaka
$$ q_0 : q_1 : q_2 : q_3 $$

\end{frame}

\begin{frame}
\noindent Ker $q_0,q_1,q_2,q_3$ niso hkrati enaki $0$, bo vsaj eno izmed zgornjih razmerij različno od $0 : 0 : 0 : 0$. Tisto razmerje nato uporabimo kot razmerje $ q_0 : q_1 : q_2 : q_3 $. Skupaj z enakostjo $q_0^2 + q_1^2 + q_2^2 + q_3^2 = 1$ nato izračunamo kvaternion $Q=(q_0,q_1,q_2,q_3)^T$ (bolj natančno sta v množici rešitev dva antipodna kvaterniona).
\subsection{Bezierjeve krivulje}
\noindent Z uporabo kinematične preslikave lahko za konstrukcijo sferičnih gibanj uporabimo Bezierjeve krivulje. Izberemo kontrolne kvaternione $Q_0,Q_1,\dots,Q_n$.
$$Q(t) = \sum\limits_{i=0}^{n} Q_i B_i^n(t)$$
\noindent Bezierjeva krivulja $Q(t)$ v času $t$ opiše kvaternion, ki mu priredimo rotacijo $R(t)$:
$$\chi(Q(t)) = R(t)$$

\noindent Rotacija, ki je določena z Bezierjevo krivuljo $Q(t) = \sum\limits_{i=0}^{n} Q_i B_i^n(t)$ stopnje $n$, je sferično razionalno gibanje stopnje $2n$. \\

\end{frame}

\begin{frame}
\noindent Gibanje koordinatnega izhodišča zapišemo v obliki
$$\underline{c}(t) = \frac{w(t)}{||Q(t)||^2};~ w(t) := (w_1(t), w_2(t),w_3(t)).$$

\end{frame}


% For every picture that defines or uses external nodes, you'll have to
% apply the 'remember picture' style. To avoid some typing, we'll apply
% the style to all pictures.
\tikzstyle{every picture}+=[remember picture]

% By default all math in TikZ nodes are set in inline mode. Change this to
% displaystyle so that we don't get small fractions.
\everymath{\displaystyle}

\begin{frame}
\frametitle{Rigid body dynamics}

\tikzstyle{na} = [baseline=-.5ex]

\begin{itemize}[<+-| alert@+>]
    \item Coriolis acceleration
        \tikz[na] \node[coordinate] (n1) {};
\end{itemize}

% Below we mix an ordinary equation with TikZ nodes. Note that we have to
% adjust the baseline of the nodes to get proper alignment with the rest of
% the equation.
\begin{equation*}
\vec{a}_p = \vec{a}_o+\frac{{}^bd^2}{dt^2}\vec{r} +
        \tikz[baseline]{
            \node[fill=blue!20,anchor=base] (t1)
            {$ 2\vec{\omega}_{ib}\times\frac{{}^bd}{dt}\vec{r}$};
        } +
        \tikz[baseline]{
            \node[fill=red!20, ellipse,anchor=base] (t2)
            {$\vec{\alpha}_{ib}\times\vec{r}$};
        } +
        \tikz[baseline]{
            \node[fill=green!20,anchor=base] (t3)
            {$\vec{\omega}_{ib}\times(\vec{\omega}_{ib}\times\vec{r})$};
        }
\end{equation*}

\begin{itemize}[<+-| alert@+>]
    \item Transversal acceleration
        \tikz[na]\node [coordinate] (n2) {};
    \item Centripetal acceleration
        \tikz[na]\node [coordinate] (n3) {};
\end{itemize}

% Now it's time to draw some edges between the global nodes. Note that we
% have to apply the 'overlay' style.
\begin{tikzpicture}[overlay]
        \path[->]<1-> (n1) edge [bend left] (t1);
        \path[->]<2-> (n2) edge [bend right] (t2);
        \path[->]<3-> (n3) edge [out=0, in=-90] (t3);
\end{tikzpicture}
\end{frame}
\end{document}